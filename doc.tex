\documentclass[a4paper]{ltxdoc}
\usepackage{hypdoc}
\hypersetup{allcolors=blue}
\usepackage[UTF8,scheme=plain]{ctex}
\IfFileExists{/System/Library/Fonts/Times.dfont}{
  \setmainfont{Times}
  \setsansfont{Helvetica}[Scale=MatchLowercase]
  \setmonofont{Menlo}[Scale=MatchLowercase]
}{\relax}
\usepackage{listings}
\lstset{basicstyle=\ttfamily,frame=single,language=bash}
\DeclareRobustCommand\file{\nolinkurl}
\DeclareRobustCommand\env{\texttt}
\DeclareRobustCommand\pkg{\textsf}
\DeclareRobustCommand\cls{\textsf}
\DeclareRobustCommand\opt{\texttt}
\renewcommand\glossaryname{版本历史}
\GlossaryPrologue{\section*{\glossaryname}}
\renewcommand\indexname{命令索引}
\EnableCrossrefs
\CodelineIndex
\RecordChanges

\begin{document}

\title{GB/T 7714-2015 \BibTeX{} style}
\author{Zeping Lee\thanks{zepinglee AT gmail.com}}
% \date{\filedate\qquad\fileversion}
\maketitle

\changes{v1.0}{2017/05/01}{Initial release.}

\section{简介}

《GB/T 7714-2015 信息与文献
参考文献著录规则》(以下简称《规则》)是我国关于参考文献的推荐标准。
宏包 \pkg{gbt-7714-2015} 是《规则》的 \BibTeX{} 实现,
包括顺序编码制和著者-出版年制两种风格。

% 本文使用以下术语:
% \begin{description}
%     \item[引用标注](citation)正文中引用的参考文献。
%     \item[著录项目](field)参考文献表中的作者、题名、出版年等项目。
% \end{description}

\section{使用方法}
宏包 \pkg{gbt-7714-2015} 应在导言区调用,如:
\begin{lstlisting}
\usepackage{gbt-7714-2015}
\end{lstlisting}

《规则》规定了顺序编码制和著者-出版年制两种风格,其中顺序编码制要求引用文献的的
序号标注在角标,而很多人希望使将序号标注在正文。为此,本宏包提供以下选项:
% \begin{description}
%     \item[super] 角标数字式顺序编码制
%     \item[numbers] 数字式顺序编码制
%     \item[authoryear] 角标数字式顺序编码制
% \end{description}


\PrintChanges
\PrintIndex
\end{document}
%
%
